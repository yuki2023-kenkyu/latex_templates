\chapter{論文の記述要領}
	\begin{enumerate}[a.]
		\item 文書作成ソフトウェア等で作成してください.なお,電子版はPDF形式で投稿してください.
		\item 用紙のサイズはA4版とし,縦長・横書きを基本とします.
		\item \cref{chap:論文の構成} の\cref{term:a}にはページ番号を付けないでください.\cref{term:b}にはローマ数字の小文字 (i,ii,...) を使ってページ番号を付けてください.\crefrange{term:c}{term:f}は,アラビア数字 (1,2,...) を使って通しでページ番号を付けてください.
		\item ページ番号は本文の下中央部に記載してください.
		\item 文章は和文 (常用漢字・現代ひらがな) あるいは英文のいずれか一方で統一して書いてください.
		\item 術語は原則として各研究室に関連の深い学会の論文執筆基準に従ってください.また人名,書名,学会名等の固有名詞は,原語のまま用いてください.

			\begin{enumerate}[(例1)]
				\item
						分かりやすい簡潔な表現[5]を心掛けることは,論文執筆において重要なことである.\\
						{[5]} 木下是雄,「理科系の作文技術」,中央公論社 中公新書624,pp.118-152,1981.
				\item
						分かりやすい簡潔な表現(5)を心掛けることは,論文執筆において重要なことである.\\
						(5) 木下是雄,「理科系の作文技術」,中央公論社 中公新書624,pp.118-152,1981.
				\item \label{例3}
						分かりやすい簡潔な表現 (木下,1981) を心掛けることは,…
						(木下,1981)  \\
						木下是雄,「理科系の作文技術」,中央公論社 中公新書624,pp.118-152 1981.\\
						※(例3)の場合,参考文献リストは,著者のABC順に,同一著者は執筆年順に並べる.
			\end{enumerate}

		\item 図表等の番号と表題は,図や写真の場合には図や写真の下に,表の場合には表の上に記載します.
		\item その他のことについては,論文の書き方を説明した \cite{木下是雄2001理科系の作文技術,学術論文の書き方・発表の仕方,howtocite} のような文献を参考にしてください.\\

	\end{enumerate}