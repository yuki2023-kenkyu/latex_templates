\documentclass{lab_thesis}

% 必要なパッケージの読み込み
\usepackage{amsmath, amssymb, graphicx}
\usepackage{hyperref} % ハイパーリンク用パッケージ

% タイトル情報の設定
\Title{これは非常に長い章タイトルであり、折り返しが必要です。タイトルが折り返された後も、続く行がタイトルの最初の行と同じ位置に揃うようにしています。}
\Author{山田 太郎}
\Professor{佐藤 一郎}
\StudentNumber{12345678}
\Date{6}

\begin{document}

% 表紙の生成
\Maketitle

% frontmatter の呼び出し
\frontmatter

% 目次、図目次、表目次の生成
\tableofcontents
\listoffigures
\listoftables

% mainmatter の呼び出し
\mainmatter

% 謝辞の挿入
\thanks{謝辞の内容をここに記述します。}

% 本文
\chapter{これは非常に長い章タイトルであり、折り返しが必要です。タイトルが折り返された後も、続く行がタイトルの最初の行と同じ位置に揃うようにしています。}
ここに序論の内容を記述します。

\section{背景}
ここに背景の内容を記述します。

\subsection{詳細な背景}
ここに詳細な背景の内容を記述します。

\chapter{短いタイトル}
ここに内容を記述します。

\section{関連研究}
ここに関連研究の内容を記述します。

\chapter{さらに長い章タイトルをここに記述して、テストを行います。これにより、すべての行が正しく整列されることを確認します。}
ここに方法論の内容を記述します。

% 付録の開始
\appendix
\chapter{付録A}
付録Aの内容を記述します。

\chapter{付録B}
付録Bの内容を記述します。

% 参考文献の挿入
\begin{thebibliography}{99}
\bibitem{sample1} 著者名, ``論文タイトル'', 雑誌名, 巻号, ページ, 年.
\bibitem{sample2} 著者名, ``論文タイトル'', 雑誌名, 巻号, ページ, 年.
\end{thebibliography}

\end{document}
