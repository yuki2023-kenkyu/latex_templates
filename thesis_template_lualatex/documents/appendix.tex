\appendix
\appendixstyle
\chapter{数値解析コード}\label{appendix:code1}
ソースコード等を付録として示す場合は以下のようにしましょう.
%ソースコードのPathはmain.tex側から見た時のPathを指定
\lstinputlisting[caption = 使用した数値解析コード, label = lst:program2]{./source_codes/source_code_1.py}
\section{その他の付録}
付録における数式番号の出力スタイルのテストです.
\begin{equation}\label{eq:test_1}
    E=mc^2
\end{equation}

\zcref{eq:test_1}は有名な式です(\zcref{appendix:code1}).

付録ソースコード参照用独自マクロのテストです.\zcref{lst:program2}はPythonのコードです.

$\SI{8}{kg}$の質量を持つ物体が,$\SI{10}{m/s}$の速度で運動しているときの運動エネルギーは$\SI{400}{J}$です.

\chapter{huhuu}

$\SI{100}{\hertz}$の周波数の音波が,$\SI{340}{\meter/\second}$の音速で伝わるときの波長は$\SI{3.4}{\meter}$です.

\cite{G_ng_r_2021,学術論文の書き方・発表の仕方,Planck:2018vyg,木下是雄2001理科系の作文技術}