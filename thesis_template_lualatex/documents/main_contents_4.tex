\chapter{論文提出について}
    \section{論文の記述要領}
        \begin{enumerate}[a.]
            \item 文書作成ソフトウェア等で作成してください.なお,電子版はPDF形式で投稿してください.
            \item 用紙のサイズはA4版とし,縦長・横書きを基本とします.
            \item 表紙には,論文題目,所属,学籍番号,氏名,指導教員名,卒業(修了)年度を記載してください(\zcref{fig:論文様式}).
            \item 修士論文を提出する場合,別途「修士論文内容要旨」(様式指定有)を提出してください.
            \item \zcref{chap:論文の構成} の\zcref{term:a}にはページ番号を付けないでください.\zcref{term:b}にはローマ数字の小文字 (i,ii,...) を使ってページ番号を付けてください.\zcref[range]{term:c,term:f}は,アラビア数字 (1,2,...) を使って通しでページ番号を付けてください.
            \item ページ番号は本文の下中央部に記載してください.
            \item 文章は和文 (常用漢字・現代ひらがな) あるいは英文のいずれか一方で統一して書いてください.
            \item 術語は原則として各研究室に関連の深い学会の論文執筆基準に従ってください.また人名,書名,学会名等の固有名詞は,原語のまま用いてください.

                \begin{enumerate}[(例1)]
                    \item
                            分かりやすい簡潔な表現[5]を心掛けることは,論文執筆において重要なことである.\\
                            {[5]} 木下是雄,「理科系の作文技術」,中央公論社 中公新書624,pp.118-152,1981.
                    \item
                            分かりやすい簡潔な表現(5)を心掛けることは,論文執筆において重要なことである.\\
                            (5) 木下是雄,「理科系の作文技術」,中央公論社 中公新書624,pp.118-152,1981.
                    \item \label{例3}
                            分かりやすい簡潔な表現 (木下,1981) を心掛けることは,…
                            (木下,1981)  \\
                            木下是雄,「理科系の作文技術」,中央公論社 中公新書624,pp.118-152 1981.\\
                            ※(例3)の場合,参考文献リストは,著者のABC順に,同一著者は執筆年順に並べる.
                \end{enumerate}

            \item 図表等の番号と表題は,図や写真の場合には図や写真の下に,表の場合には表の上に記載します.
            \item その他のことについては,論文の書き方を説明した \cite{木下是雄2001理科系の作文技術,学術論文の書き方・発表の仕方,howtocite} のような文献を参考にしてください.\\

        \end{enumerate}
        修士論文審査基準等に関する詳細は,福島大学「修士論文に関する取扱要項」\url{https://www.fukushima-u.ac.jp/Files/2020/05/sssm.pdf}を参照してください.

    \figimage{./images/論文様式.png}{論文表紙の様式}{fig:論文様式}{0.5}

    \section{必要部数}
        \begin{itemize}
            \item 卒業論文:冊子体(様式指定なし)1部および電子版 (PDFファイル)
            \item 修士論文:冊子体(フラットファイルに閉じたもの)3部\footnote{論文審査委員が4名以上の場合は,その人数分の部数を準備すること.}および電子版 (PDFファイル)
        \end{itemize}

    \section{提出先}
    卒業論文,修士論文は指導教員に論文を提出し確認を受けた後,教務課窓口に期日までに冊子体を提出する.修士論文の場合は,副査の先生方にメール等で事前にアポイントメントを取り,提出様式(メールで電子版を提出する,もしくは紙媒体かなど)を確認し,指定された日までに提出する(先生方に相談の上教務課提出後でもよい).
        \begin{itemize}
            \item 卒業論文:指導教員,教務課窓口
            \item 修士論文:主査(指導教員),教務課窓口,副査
        \end{itemize}

    \section{最終版の提出}
    卒業論文発表会,修士論文最終試験を終えて,修正等を行った後,論文最終版を提出する.
        \begin{itemize}
            \item 卒業論文:研究室に保存するため\url{https://forms.gle/jXTzupgmt3pjyxqN7}にPDFをアップロード.
            \item 修士論文:教務課に修士論文内容要旨とともに最終版を期日までに提出,研究室に保存するため\url{https://forms.gle/jXTzupgmt3pjyxqN7}にPDFをアップロード.
        \end{itemize}