\documentclass{japanese_thesis}
\usepackage{./preambles/packages_lualatex}
\usepackage{./preambles/macros}
\addbibresource{./references/reference_1.bib}

\title{サンプル文書}
\author{あなたの名前}
\date{\today}

\begin{document}

\maketitle
これはサンプル文書です。文献を参照します。
\ajRoman{1}a型超新星

作用は、
\begin{eqnarray}\label{eq:EH_action}
    S=\int d^4x\sqrt{-g}\left[\frac{1}{2\kappa}R+\lagrd _M \right]
\end{eqnarray}
で与えられる。計量の逆数の変分$\delta{g^{\mu\nu}}$を計算すると、
\begin{eqnarray}
    \delta{S}&=&\int d^4x \delta{g^{\mu\nu}}\left[\frac{1}{2\kappa}\fdv{(\sqrt{-g}R)}{g^{\mu\nu}}+\fdv{(\sqrt{-g}\lagrd_M)}{g^{\mu\nu}} \right]\\
    &=&\int d^4x \delta{g^{\mu\nu}}\left[\frac{1}{2\kappa}\left(\fdv{R}{g^{\mu\nu}}+\frac{R}{\sqrt{-g}}\fdv{\sqrt{-g}}{g^{\mu\nu}}\right)+\frac{1}{\sqrt{-g}}\fdv{(\sqrt{-g}\lagrd_M)}{g^{\mu\nu}} \right]\label{eq:EH_var}
\end{eqnarray}

\begin{eqnarray}
    \odv{}{t}\left(\pdv{L}{\dot{q}_i}\right)-\pdv{L}{q_i}=0
\end{eqnarray}

\nocite{*}
\printbibliography[title=参考文献]

\end{document}
